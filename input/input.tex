\documentclass[a4paper,10pt]{ltjsarticle}

\usepackage{_my_style}
\renewcommand{\labelitemi}{・}
\renewcommand{\labelitemii}{・}

\begin{document}

% title
% author
% date
\begin{center}
  {\LARGE 入力情報の整理}\
  \vskip6.5pt
  {\large たけうち}\
  \vskip2pt
  {\large }
\end{center}
\section*{入力まとめ}
入力は以下のように与えられます.
\begin{equation*}
  \begin{array}{ccc}
    L & N &  \\
    T_1 &  & \\
    t_{1,1} & b_{1,1} & m_{1,1} \\
    \vdots && \\
    t_{1, T_1} & b_{1, T_1} & m_{1, T_1}\\
    \vdots \\
    T_N \\
    t_{N,1} & b_{N,1} & m_{N,1}\\
    \vdots \\
    t_{N, T_N} & b_{N, T_N} & m_{N, T_N}
  \end{array}
\end{equation*}


$1$ 行目は,最初の勇者の強さ$L$,タワーの数$N$が与えられます.

続く$2$行目に,$1$つ目のタワーの高さが与えられます.以下に続く$T_1$行の上から$i\ (1\leq i \leq T_1)$行には,$i$階の情報$t,b,m$が与えられます.

この下に,同様のものが$N-1$個に続いています.

\subsection*{入力制限}

入力はすべて$1$以上の整数です.

\section*{フロアの情報まとめ}
フロアの情報が$t,b,m$であるような場所に勇者が行ったとき,勇者の強さの変動が記されています.
\begin{itemize}
  \item $t=1$のとき,敵がいることを表します.このとき,勇者の強さが$m$未満の場合無効となります.
    \begin{itemize}
      \item $b = 1$ のとき,勇者の強さは$+m$されます.
      \item $b = 2$ のとき,勇者の強さは$\times m$されます.
    \end{itemize}
  \item $t=2$のとき,装備や薬があることを表します.
    \begin{itemize}
      \item $b = 1$のとき,勇者の強さは$+m$されます.
      \item $b = 2$のとき,勇者の強さは$-m$されます.ここで,勇者の強さが$0$以下となった場合無効となります.
      \item $b = 3$ のとき,勇者の強さは$\times m$されます.
      \item $b = 4$のとき,勇者の強さは$// m$されます.($m$で割ったときの商)ここで,勇者の強さが$0$以下となった場合無効となります.
    \end{itemize}
\end{itemize}


\end{document}